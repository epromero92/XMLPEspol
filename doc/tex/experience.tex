\documentclass[../main.tex]{subfiles}
\begin{document}
\section{Experiencia usando Python}
\begin{itemize} 
\item Me agrada mucho el hecho de que python es un lenguaje multiparadigma, ya que aunque es posible programar usando el paradigma estructurado, orientado a objetos o el paradigma funcional.
\item Considero que la sintaxis es bastante simple y clara, lo cual facilita la lectura, escritura de código y el aprendizaje.
\item Python es un lenguaje interpretado y el intérprete del lenguaje está disponible para varias plataformas, lo cual es un gran punto a favor para la portabilidad.
\item Debido a que en este lenguaje no se realizan declaraciones explícitas de tipos, pr\'acticamente no es necesario realizar sobrecarga de m\'etodos, ya que los tipos de variables y funciones estar\'an determinados por las asignaciones realizadas. Esta cualidad, aunque es un punto a favor de la expresividad/facilidad de escritura, puede resultar un poco confusa (detrimento para la legibilidad) si se trata de funciones con muchos bloques anidados y adem\'as pueden resultar problemas de seguridad.
\item Considero que la simplicidad y versatilidad que poseen las estructuras b\'asicas como: listas, tuplas y diccionarios realmente facilita la tarea de representar objetos complejos.
\item Estoy acostumbrado a usar la indentaci\'on aunque no sea un requerimiento del lenguaje, ya que pienso que mantiene el c\'odigo organizado y facilita ampliamente la lectura y comprensión del mismo, por esta razón considero que el hecho de que la indentación sea obligatoria en este lenguaje, es un punto a favor del mismo.
\end{itemize}
\paragraph{ }
Conclusión: Pienso que el estudio de nuevos lenguajes amplía en gran cantidad la capacidad para expresar nuestras ideas, por esta razón y por todo el potencial que veo en este lenguaje, considero que profundizar en el estudio de Python es una excelente decisión.
\end{document}